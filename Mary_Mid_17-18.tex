\documentclass[a4paper]{article} % this is used for comments
\usepackage[utf8]{inputenc}
%%% Дополнительная работа с математикой
\usepackage{amsmath,amsfonts,amssymb,amsthm,mathtools} % AMS
\usepackage{icomma} % "Умная" запятая: $0,2$ --- число, $0, 2$ --- перечисление
\usepackage[english,russian]{babel}

%% Номера формул
\mathtoolsset{showonlyrefs=true} % Показывать номера только у тех формул, на которые есть \eqref{} в тексте.

%% Шрифты
\usepackage{euscript}	 % Шрифт Евклид
\usepackage{mathrsfs} % Красивый матшрифт

\usepackage{physics}

%% Свои команды
\DeclareMathOperator{\sgn}{\mathop{sgn}}

%% Перенос знаков в формулах (по Львовскому)
\newcommand*{\hm}[1]{#1\nobreak\discretionary{}
{\hbox{$\mathsurround=0pt #1$}}{}}

\DeclareMathOperator{\Lin}{\mathrm{Lin}}
\DeclareMathOperator{\Linp}{\Lin^{\perp}}
\DeclareMathOperator*\plim{plim}
\DeclareMathOperator{\grad}{grad}
\DeclareMathOperator{\card}{card}
\DeclareMathOperator{\sgn}{sign}
\DeclareMathOperator{\sign}{sign}

\DeclareMathOperator*{\argmin}{arg\,min}
\DeclareMathOperator*{\argmax}{arg\,max}
\DeclareMathOperator*{\amn}{arg\,min}
\DeclareMathOperator*{\amx}{arg\,max}
\DeclareMathOperator{\cov}{Cov}
\DeclareMathOperator{\Var}{Var}
\DeclareMathOperator{\Cov}{Cov}
\DeclareMathOperator{\Corr}{Corr}
\DeclareMathOperator{\pCorr}{pCorr}
\DeclareMathOperator{\E}{\mathbb{E}}
\let\P\relax
\DeclareMathOperator{\P}{\mathbb{P}}

\newcommand{\cN}{\mathcal{N}}
\newcommand{\cU}{\mathcal{U}}
\newcommand{\cBinom}{\mathcal{Binom}}
\newcommand{\cBin}{\cBinom}
\newcommand{\cPois}{\mathcal{Pois}}
\newcommand{\cBeta}{\mathcal{Beta}}
\newcommand{\cGamma}{\mathcal{Gamma}}

\newcommand \R{\mathbb{R}}
\newcommand \N{\mathbb{N}}
\newcommand \Z{\mathbb{Z}}

\newcommand{\dx}[1]{\,\mathrm{d}#1} % для интеграла: маленький отступ и прямая d
\newcommand{\ind}[1]{\mathbbm{1}_{\{#1\}}} % Индикатор события
%\renewcommand{\to}{\rightarrow}
\newcommand{\eqdef}{\mathrel{\stackrel{\rm def}=}}
\newcommand{\iid}{\mathrel{\stackrel{\rm i.\,i.\,d.}\sim}}
\newcommand{\const}{\mathrm{const}}

% вместо горизонтальной делаем косую черточку в нестрогих неравенствах
\renewcommand{\le}{\leqslant}
\renewcommand{\ge}{\geqslant}
\renewcommand{\leq}{\leqslant}
\renewcommand{\geq}{\geqslant}





\title{Промежуточный экзамен 2017-2018}
\author{БЭК182, Шушпанова Мария}
\date{Июнь 2020}






\begin{document}

\maketitle

\textbf{Промежуточный экзамен 2017-2018}

\textbf{Ответы:}
AEСBA ?BCEB BBCCA BBCCC ABCBA EA?AC

\vspace{\baselineskip}
\textbf{Решение:}

\begin{enumerate}

    \item

    $X^2$ - неотрицательная случайная величина. 
    
    Тогда согласно неравенству Маркова:

    \[\P(X^2 \geq 100) \leq \frac{\E(X^2)}{100} \]
    
    Находим необходимое математическое ожидание и вычисляем ответ:
     \[ (\E(X^2) = \Var(X) + \E(X))^2 = 10\]
     \[\P(X^2 \geq 100) \leq \frac{10}{100} = 0.1 \]
    
    Следовательно, $\P(X^2 \geq 100)$ принадлежит промежутку $[0, 0.1]$

    Ответ: A

    \item

    По определению распределения Пуассона с параметром $\lambda$:
    \[ \E(\xi) =\lambda \]
    \[ \Var\xi) =\lambda \]
    
    Тогда по свойству дисперсии:

    \[ \E(\xi^2) = \Var(\xi) + (\E(\xi))^2 = \lambda + \lambda^2 = \lambda \cdot (1 + \lambda)\]

    Ответ: E

    \item
    
    Вычислим неизвестные дисперсию и ковариацию:
    \[ \Var(X+Y) = \Var(X) + \Var(Y) + 2 \cdot \Cov(X,Y) = 4 + 9 + 2 \cdot (-3) = 7 \]
    \[\Cov(X+Y,Y) = \Cov(X,Y) + \Cov(Y,Y) = -3 + 9 = 6\]
    
    
    Тогда по формуле корреляции:
    \[ \Corr(X+Y,Y) = \frac{\Cov(X+Y,Y)}{\sqrt{\Var(X+Y)\cdot \Var(Y)}} = \frac{6}{\sqrt{63}} = \frac{6}{3\sqrt{7}} = \frac{2}{\sqrt{7}} \]

    Ответ: C

    \item
    
    Функция плотности нормально распределенной случайной величины:
    
    \[ f(x)=\frac{1}{\sqrt{2\pi\sigma^2}} e^{-\frac{(x-\mu)^2}{2\sigma^2}} \]
    
    Известно, что для стандартного нормального распределения $\mu=0$, а $\sigma^2=1$. Тогда его функция плотности:
    
    \[ f(x)=\frac{1}{\sqrt{2\pi}} e^{-\frac{x^2}{2}} \]
    
    Для вычисления искомой вероятности нужно проинтегрировать полученное выражение на промежутке [-1; 2]
    
    Ответ: B

    \item

    Координаты вершин соответсвующего треугольника: $(0;0), (2;0), (0;4)$
    Тогда его площадь $S_\triangle$ = 4
   
    Для равномерного распределения:
    
    \[f_{X,Y}(1,1) = \frac{1}{S_\triangle} = \frac{1}{4}\]

    Ответ: A

    \item
    
    По определению события A, B и C независимы в совокупности, если:
    
    \[ \P(A \cap B) = \P(A)\P(B) \]
    \[ \P(A \cap C) = \P(A)\P(C) \]
    \[ \P(B \cap C) = \P(B)\P(C) \]
    \[ \P(ABC) = \P(A)\P(B)\P(C) \]

    Ответ: B, D

    \item
    
    Для определения искомой вероятности вычислим интеграл:
    
    \[\P({\xi\in[3; 6]}) = \int_3^4 \frac{1}{4} dx = \left.\frac{x}{4}\right|_3^4  = 1 - \frac{3}{4} = \frac{1}{4}\]

    Ответ: B

    \item

    Заметим, что для каждого значения с.в. $Y$ существует только одно значение с.в. $X$, удовлетворяющее заданному условию: 
    
    \[Y = -1; X = 1\]
    \[Y = 0; X = 2\]
    \[Y = 1; X = 1\]
    
    Тогда искомая вероятность равна сумме вероятностей этих событий.
    Вследствие независимости данных сулчайных величин получаем:

    \[\P(X + Y^2 = 2) = (\frac{1}{11} \cdot\frac{1}{3}) \cdot 3 = \frac{1}{11}\]

    Ответ: C

    \item
    
    Длина окружности с единичным радиусом равна $2\pi$. 
    Тогда всего Вася может попасть в один из шести секторов 
    
    По классическому определению вероятности:

    \[\P(\text{<<Красный>>}) = \frac{1}{6}\]

    Ответ: E

    \item
    
    По теореме сложения:

    \[\P(A\cup B) = \P(A) + \P(B) - \P(A\cap B)\]
    \[\P(B) = \P(A\cup B) + \P(A\cap B)\ - \P(A) = 0.6 + 0.2 - 0.3 = 0.5\]

    Ответ: B

    \item
    
    По свойствам дисперсии:

    \[\Var(2X - Y + 1) = 4\cdot \Var(X) + \Var(Y) - 4 \cdot \Cov(X,Y) \]
    \[\Var(2X - Y + 1) = 4 \cdot 4 + 9 - 4 \cdot (-3) = 37\]

    Ответ: B

    \item
    
    По закону больших чисел данный предел равен $\E(X^2)$
    В свою очередь для стандартного нормального распределения:

    \[\E(X^2) = \Var(X) +(\E(X))^2 = 1 + 0 = 1\]

    Ответ: B

    \item

    Из определения условной функции плотности с.в. X при Y = 1/2:

    \[f\left(x\mid y=\frac{1}{2}\right) = \frac{f\left(x,\frac{1}{2}\right)}{f_{y}\left(\frac{1}{4}\right)} = \frac{6x \cdot 0.25}{0.75} = 2x\]

    \[f_{y}(y) = \int_0^1 6\cdot x\cdot y^2 dx = \left.3 \cdot y^2 \cdot x^2\right|_0^1  = 3\cdot y^2,  y \in [0;1]\]

    Ответ: C

    \item

    Из искомой вероятности устанавливаем, что $n$ = 100
    Тогда:

    \[\Bar X \sim \N\left(4,1\right)\]
    
    Стандартизируем случайную величину и, пользуясь таблицей, получаем ответ:

    \[\P\left(\frac{\Bar X - 4}{\sqrt{1}} \leq \frac{5-4}{1}\right) = \P(\Z\leq 1) = 0.84$$
    
    Ответ: C

    \item
    
    По свойствам ковариации:

    \[\Cov(X+2Y, 2X + 3) = \Cov(X, 2X) + \Cov(2Y, 2X) = 2\Var(X) + 4 \cdot \Cov(X,Y)\] 
    \[\Cov(X+2Y, 2X + 3) = 2 \cdot 4 + 4 \cdot (-3) = 8 - 12 = -4\]

    Ответ: A

    \item
    
    По свойствам математического ожидания:

    \[\E((X-1)Y) = \E(XY) - \E(Y) = \E(X)\cdot\E(Y) + \Cov(X,Y) - \E(Y)\] 
    \[\E((X-1)Y) = - 2 - 3 - 2 = -7\]

    Ответ: B

    \item
    
    Случайная величина $X_{i}$ имеет распределение Бернулли. 
    
    Причем $\P(X_{i} = 1) = \frac{1}{6}$
    
    Тогда:

    \[\P(X_{1} + X_{2} = 1) = \P(X_{1} = 0; X_{2} = 1) + \P(X_{1} = 1; X_{2} = 0) \]
    \[\P(X_{1} + X_{2} = 1) = \frac{5}{6} \cdot \frac{1}{6} + \frac{1}{6} \cdot \frac{5}{6} = \frac{5}{18}\]
    
    \[\P(X_{1} = 0\mid X_{1} + X_{2} = 1) = \frac{\P(X_{1} = 0\cap X_{1} + X_{2} = 1)}{X_{1} + X_{2} = 1}) = \frac{1}{2}\]

    Тогда найденный условный закон распределения случайной величины $X_{1}$ совпадает с распределением Бернулли с параметром $p = \frac{1}{2}$
    
    Ответ: B

    \item

    Стандартизируя случайную величину $X$ + $Y$ получаем:

    \[\P(X + Y < 3) = \P\left(\frac{X+Y-3}{\sqrt{7}} < \frac{3-3}{\sqrt{7}}\right) = (\Z\leq 0) = 0.5\]

    Ответ: C

    \item

    Существуют 3 доступные функции кнопок:

    Честные кубики (\P(x_i = 6) =\frac{1}{2}$) 

    Увеличенная вероятность шестерки (\P(x_i = 6) =\frac{1}{2}$) 

    Увеличенная вероятность единицы (\P(x_i = 6) =\frac{1}{10}$) 
    
    Тогда:

    \[\P(\text{<<Честный кубик>>}\mid \text{<<6>>}) = \frac{\P(i=1,2,3\cap \text{<<6>>})}{\P(\text{<<6>>})} = \frac{\frac{3}{30}}{\frac{11}{50}} = \frac{5}{11} \]

    Ответ: C

    \item

    Из всех данных матриц только матрица ""С"" обладает свойствами симметричности и неотрицательности определителя

    Ответ: C

    \item
    
    По свойствам математического ожидания:

    \[\E(\alpha X + (1 - \alpha)Y) = \alpha \E(X) + (1-\alpha) \E(Y) = \alpha \cdot (-1) + (1-\alpha)\cdot2 = 0\]
    \[\alpha = \frac{2}{3}\]

    Ответ: A

    \item
    
    Для биноминального распределения:

    \[\P(\xi = 0) = (1-p)^n = (\frac{1}{4})^2 = \frac{1}{16}\]

    Ответ: B

    \item
    
    Для распределение Пуассона с параметром $\lambda = 4$:
    
    \[\P(x=k) = \lambda^k\cdot\frac{e^{-\lambda}}{k!}\]
    \[\P(X\geq1) = 1 - \P(k=0) = 1 - e^{-4}\]
    
    Обратите внимание, что приведенная формула работает только для маленьких вероятностей
    
    Ответ: C

    \item
    
    Для распределения Бернулли с параметром p:

    \[\E(\xi^2) = \Var(\xi) + (\E(\xi))^2 = p\cdot (1-p) + p^2 = p\]

    Ответ: B

    \item
    
    Для экспоненциального распределения:

    \[\E(\xi) = \frac{1}{\lambda}\]
    \[\Var(\xi) = \frac{1}{\lambda^2}\]

    \[\E(\xi^2) = \Var(\xi) + (\E(\xi))^2 = \frac{1}{\lambda^2} + \frac{1}{\lambda^2} = \frac{2}{\lambda^2}\]

    Ответ: A

    \item

    При всех стараниях Васе никак не удастся попасть дротиком одновременно и в красный, и в синий сектор, поэтому события А и В несовместны
    
    Ответ: E

    \item
    
    Для нахождения искомого м.о. вычислим соответствующий интеграл:
    
    \[\E(XY) = 6 \cdot \int_0^1 dx\int_0^1 x\cdot y \cdot x \cdot y^2 dy = \left.\int_0^1 2\cdot x^3 \cdot y^3 \right|_0^1 dy = \left.\frac{2\cdot y^4}{4}\right|_0^1 = \frac{1}{2}\]

    Ответ: A

    \item
    
    По свойствам дисперсии:
    \[\Var(\alpha X + (1-\alpha) Y) = \alpha^2 \Var(X) + (1-\alpha)^2 \Var(Y) + 2\cdot \Cov(X,Y)\cdot\alpha\cdot(1-\alpha)\]
    \[\Var(\alpha X + (1-\alpha) Y) = 4\cdot\alpha^2 + 9\cdot(1-\alpha)^2 - 6\cdot\alpha\cdot(1-\alpha) = 19\cdot\alpha^2 - 24\cdot\alpha + 9\]

    Тогда точка минимума параболы, ветви которой направленны вверх:
    \[\alpha = \frac{24}{2 \cdot 19} =\frac{12}{19}\]

    Ответ: 12/19

    \item
    
    Для ответа на вопрос вычислим ряд условных вероятностей:

    \[\P(\text{<<без багажа>>}) = \frac{1}{4}\]
    \[\P(\text{<<с рюкзаком>>}\mid\text{<<без багажа>>}) = \frac{1}{2}\]
    \[\P(\text{<<с рюкзаком>>}\mid\text{<<c багажом>>}) = \frac{55}{150}\]
    
    Тогда \P(\text{<<без рюкзака>>}):
    
    \[\P(\text{<<без рюкзака>>}) = \frac{1}{2}\cdot\frac{1}{4} + \frac{95}{150} \cdot\frac{3}{4} = 0.6\]

   

    Ответ: A

    \item
    
    Cогласно неравенству Маркова:
   
    \[\P(|X-2| \geq 10) \leq \frac{\Var(X)}{100}\]
    
    Тогда искомая вероятность:

    \[\P(|X-2| \leq 10) \geq 1 - \frac{\Var(X)}{100} = 0.94\]

    Ответ: C

\end{enumerate}

\end{document}